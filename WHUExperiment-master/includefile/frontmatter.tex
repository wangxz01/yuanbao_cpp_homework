% !Mode:: "TeX:UTF-8"

%%% 此部分需要自行填写: 中文摘要及关键词 

%%% 郑重声明部分无需改动

%%%---- 郑重声明 (无需改动)------------------------------------%
\newpage
\thispagestyle{empty}
\vspace*{20pt}
\begin{center}{\ziju{0.8}\pmb{\songti\zihao{2} 郑重声明}}\end{center}
\par\vspace*{30pt}
\renewcommand{\baselinestretch}{2}

{\zihao{4}%

本人呈交的设计报告,是在指导老师的指导下,独立进行实验工作所取得的成果,
所有数据、图片资料真实可靠. 尽我所知,除文中已经注明引用的内容外,
本设计报告不包含他人享有著作权的内容.
对本设计报告做出贡献的其他个人和集体,
均已在文中以明确的方式标明.本设计报告的知识产权归属于培养单位.\\[2cm]

\hspace*{1cm}本人签名: $\underline{\hspace{3.5cm}}$
\hspace{2cm}日期: $\underline{\hspace{3.5cm}}$\hfill\par}
%------------------------------------------------------------------------------
\baselineskip=23pt  % 正文行距为 23 磅
%------------------------------------------------------------------------------





%%======摘要===========================%
\begin{cnabstract}
\thispagestyle{empty}

元胞自动机(Cellular Automaton, CA)是一种离散模型,它由一个格子阵列和定义在格子上的状态组成.元胞自动机的每一个格子可以处于有限数量的状态,格子的状态在离散的时间步骤内根据一组规则进行更新.元胞自动机广泛应用于物理学、生物学、计算机科学等领域.

在本次实验中中,主要完成了1.部分元胞自动机的控制台展现代码实现;2.利用sdl库在mac平台上实现图形界面;3.相关分析和介绍.

由于网络上流传的代码大都是在windows系统上实现,需要使用到windows.h库,作为在mac上使用sdl库实现图形化的一个成果,还是具有一定的跨平台优越性的.
\end{cnabstract}
\par
\vspace*{2em}

%%%%--  关键词 -----------------------------------------%%%%%%%%
%%%%-- 注意: 每个关键词之间用“;”分开,最后一个关键词不打标点符号
\cnkeywords{元胞自动机	;  SDL;  C++}